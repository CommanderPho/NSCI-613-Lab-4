\documentclass[12pt]{article}
\usepackage[usenames]{color} %used for font color
\usepackage{amssymb}
\usepackage{pho-equation}
\usepackage[assignment]{pho-format}

%\usepackage{url} % Formatting packages
\usepackage{hyperref} % Required for adding links and customizing them
\definecolor{linkcolour}{rgb}{0,0.2,0.6} % Link color
\hypersetup{colorlinks,breaklinks,urlcolor=linkcolour,linkcolor=linkcolour} % Set link colors throughout the document


\newcommand{\githubRepoNameFormat}[1]{ \textsl{\footnotesize#1} }
%\github{real-URL}{display-name}
%\newcommand{\github}[2]{\faGithub\href{#1}{#2}}
\newcommand{\github}[2]{\href{#1}{\githubRepoNameFormat{#2}}}

%\githubFromShortName{githubUser}{repoName}{displayName}
\newcommand{\githubFromShortName}[3]{\github{https://github.com/#1/#2}{#3}}

%\phoGithub{repoName}{displayName}
%\newcommand{\phoGithub}[2]{\githubFromShortName{CommanderPho}{#1}{#2 (Private)}}
\newcommand{\phoGithub}[2]{\githubFromShortName{CommanderPho}{#1}{#2}}



\title{NSCI 613 - Lab 4}
\author{\phoName}
\subject{ NSCI 613 - Neurophysiology and Computational Neuroscience }
%\studentID{2437149}


\begin{document}
\maketitle
\vfill
All source code for this assignment is available in my Github repository: \phoGithub{NSCI-613-Lab-3}{https://github.com/CommanderPho/NSCI-613-Lab-3}

\updateheaders
\clearpage

%\slntParagraph{ Increasing the delay of the second current pulse from $9[ms]$ up in $0.1[ms]$ increments, we find that a delay of approximately $10.3[ms]$ is the minimum refractory time between action potential events evoked by a current pulse of the same magnitude. See Figure \ref{fig:P1} for the voltage curve.}


%\begin{figure}[H]
%\centering
%\includegraphics[width=0.9\textwidth]{Results/P1}
%\caption{\label{fig:P1} Voltage Plots used to determine the minimum refractory time between action potential events for P1. }
%\end{figure}

\newcommand*\mgate{$\mathcolor{charcoal}{\codeStyleVar{vshift\_m}}$}

\newcommand*\hgate{$\mathcolor{R_d}{\codeStyleVar{vshift\_h}}$}

\newcommand*\ngate{$\mathcolor{B_l}{\codeStyleVar{vshift\_n}}$}

% P1
\Question{Neurons in Action - Na and K Channel Kinetics tutorial}

\begin{given}
	\gitem{ $\tdefn{\mgate{}}{Na+ Activation Gate}$ }
	\gitem{ $\tdefn{\hgate{}}{Na+ Inactivation Gate}$ }
	\gitem{ $\tdefn{\ngate{}}{K+ Activation Gate}$ }
\end{given}


%%%%%%%%%% Nav 1.6 Subtype
\section*{Nav 1.6 subtype: $\mgate{} = 14\utext{mV}$ , $\hgate{} = 10 \utext{mV}$ }

%% P1.A - Nav1.6
\subQuestion{Determining threshold amplitude that generates an AP for Nav1.6}


\begin{figure}[H]
\centering
\includegraphics[width=0.8\textwidth]{Results/1a-m14-h10-n0-0528}
\caption{\label{Fig:1a0}Plots for $\mgate{} = 14 \utext{mV}$, $\hgate{} = 10 \utext{mV}$, $\ngate{} = 0 \utext{mV}$ used to determine the new threshold current amplitude of $0.528 \utext{nA}$ for Nav 1.6.}
\end{figure}

\par The smallest amplitude of current that could trigger an AP for these parameters was \sln{$0.528 \utext{nA}$} which is plotted in Figure \ref{Fig:1a0}.



%% P1.B - Nav1.6
\subQuestion{Gating variables and inactivation functions for Nav1.6}

\begin{figure}[H]
\centering
\includegraphics[width=0.8\textwidth]{Results/1a-default-Na-Kinetics-m0-h0-n0-011}
\caption{\label{Fig:1a-ref-011}Plots for the default Na+ channel kinetics ($\mgate{} = 0 \utext{mV}$, $\hgate{} = 0 \utext{mV}$, $\ngate{} = 0 \utext{mV}$) stimulated at the threshold current amplitude of $0.11 \utext{nA}$.}
\end{figure}

\slntParagraph{Comparing the Nav 1.6 subtype (Figure \ref{Fig:1a0}) with the default Na+ channel kinetics (Figure \ref{Fig:1a-ref-011}) we can see that the sodium activation gating variable $\mgate{}$ curve was shifted to the right, meaning it would be less open/active for a given voltage. This manifests on our current plot as a notably slower rise in the sodium current. The sodium inactivation gating variable \hgate{} was also shifted in the to the right (in the positive direction), meaning lower levels of sodium inactivation at rest (See the high t=0 value of the red line in the second subplot from the bottom in Figure \ref{Fig:1a0}) as well as a delayed inactivation of sodium channels were observed for the same potentials. This led to increased amplitude of APs, as well as a longer time spent at higher membrane voltages. It was primarily the slower sodium activation dynamics that competed with the K+ current to determine if an AP was fired. More current had to be applied to attain an AP in the Nav 1.6 case to compensate for the slower Na+ activation to compete with the same potassium current. }

\begin{tabular}{l|cr}
Gate Var. & Shift Dir & Consequence \\ \hline
\mgate{} & Right & Less activation/open channels for the same voltage \\
\hgate{} & Right & Reduced/delayed inactivation of Na+ channels \\
\ngate{} & - & -
\end{tabular}


%% P1.C - Nav1.6
\subQuestion{Attempting to compensate with K+ channel for Nav1.6}

\slntParagraph{No, it does not appear to be possible to induce an AP by adjusting $\ngate{}$ for a stimulation current of $0.11 \utext{nA}$.}

\begin{figure}[H]
\centering
\includegraphics[width=0.8\textwidth]{Results/1c-m14-h10-n15-0235}
\caption{\label{Fig:1c0}Plots for $\mgate{} = 14 \utext{mV}$, $\hgate{} = 10 \utext{mV}$, $\ngate{} = 15 \utext{mV}$.}
\end{figure}

%TODO: Finish "Why" part of question.
\par The lowest threshold current amplitude possible appears to be \slntParagraph{$0.235 \utext{nA}$ with $\ngate{} = 15$} This is shown in Figure \ref{Fig:1c0}.




\clearpage
%%%%%%%%%% Nav 1.1 Subtype
\section*{Nav 1.1 subtype: $\mgate{} = 7 \utext{mV}$, $\hgate{} = -9.5 \utext{mV}$}
% Reset the counter as a hack so "a", "b", "c" are generated again:
\setcounter{subQuestionCounter}{0} 


%% P1.A - Nav1.1
\subQuestion{Determining threshold amplitude that generates an AP for Nav1.1}


\begin{figure}[H]
\centering
\includegraphics[width=0.8\textwidth]{Results/1a-m7-h95-n0-06}
\caption{\label{Fig:1a1} Plots for $\mgate{} = 7 \utext{mV}$, $\hgate{} = -9.5 \utext{mV}$, $\ngate{} = 0 \utext{mV}$ used to determine the threshold current amplitude of $0.6 \utext{nA}$ for Nav 1.1.}
\end{figure}

\par The smallest amplitude of current that could trigger an AP for these parameters was \sln{$0.6 \utext{nA}$} where it reaches $11.2 \utext{mV}$ at its peak. This is plotted in Figure \ref{Fig:1a1}.



%% P1.B - Nav1.1
\subQuestion{Gating variables and inactivation functions for Nav1.1}

\slntParagraph{Comparing the Nav 1.1 subtype (Figure \ref{Fig:1a1}) with the default Na+ channel kinetics (Figure \ref{Fig:1a-ref-011})  we can see that the sodium activation gating variable $\mgate{}$ curve was again shifted to the right, leaving it less open/active for a given voltage. Unlike the Nav 1.6 case, the sodium inactivation gating variable \hgate{} was shifted in the opposite direction (to the left, in the negative direction). This resulted in faster inactivation of the sodium channels than in the default kinetics case, meaning even more sustained applied current was required to reach an AP than for the Nav 1.6 subtype. }

\begin{tabular}{l|cr}
Gate Var. & Shift Dir & Consequence \\ \hline
\mgate{} & Right & Less activation/open channels for the same voltage \\
\hgate{} & Left & Higher/Sooner inactivation of Na+ channels \\
\ngate{} & - & -
\end{tabular}


%% P1.C - Nav1.1
\subQuestion{Attempting to compensate with K+ channel for Nav1.1}

%TODO: Finish "Why" part of question.
\slntParagraph{Yes adjusting $\ngate{} = 45 \utext{mV}$ induces an AP with a peak around $10 \utext{mV}$ for an applied stimulation current of $0.11 \utext{nA}$.}


\begin{figure}[H]
\centering
\includegraphics[width=0.8\textwidth]{Results/1c-m7-h95-n45-011}
\caption{Plots for $\mgate{} = 7 \utext{mV}$, $\hgate{} = -9.5 \utext{mV}$, $\ngate{} = 45 \utext{mV}$.}
\end{figure}


%% Defaults:


\begin{figure}[H]
\centering
\includegraphics[width=0.8\textwidth]{Results/1a-default-Na-Kinetics-m0-h0-n0-0528}
\caption{Plots for $\mgate{} = 0 \utext{mV}$, $\hgate{} = 0 \utext{mV}$, $\ngate{} = 0 \utext{mV}$ stimulated at the higher current amplitude of $0.528 \utext{nA}$, which is the threshold current amplitude for the Nav1.6 Subtype.}
\end{figure}





% P2
\Question{MATLAB Golomb Neuron Model - M-type K+ current and Spiking Frequency}

\begin{figure}[H]
\centering
\includegraphics[width=0.9\textwidth]{Results/2-1}
\caption{\label{Fig:2a1} Membrane Voltage and M-Type K+ Current plots for several important values of gM that mark a transition in firing dynamics. From top to bottom, we have gM = {0.0, 0.4, 0.5, 1.2, 1.3, 1.5}. }
\end{figure}

\begin{figure}[H]
\centering
\includegraphics[width=0.6\textwidth]{Results/2-2}
\caption{\label{Fig:2a2} Membrane Voltage Curve Spiking Frequency as a function of M-Type K+ Conductance gM. }
\end{figure}

% gM Voltage Spiking
\begin{table}[H]                                                                         
\centering                                                                            
\begin{tabular}{|l|l|l|}                                                              
\hline                                                                                
gM & Membrane Voltage Curve Observations & $\text{Freq}_{\text{spike}}$ \\                                           
\hline                                                                                
0.000 & High frequency single spike firing & 135.156 \\                                                          
\hline                                                                                
0.400 & Frequency of spikes has been decreasing & 28.509 \\                                                           
\hline                                                                                
0.500 & Spikes transition to occurring in bursts of two & 14.799 \\
\hline                                                                                
0.600 & Pairs of spikes decrease in frequency (spread out) & 26.332 \\
\hline
0.700 & Spikes have spread out to the extent that it looks similar to single-spike firing & 11.044 \\
\hline                                                                                
1.200 & Spikes are very low frequency (highly spaced out) & 6.312 \\                                                            
\hline                                                                                
1.300 & Spiking is abolished. After the first spike, only a subthreshold transient remains. & - \\                                                                
\hline                                                                                
\end{tabular}                                                                         
\caption{Membrane Voltage Curve Spiking Frequencies at the Transition gM values.}
\label{table:2a3}                                                                 
\end{table}                                                                           

% gM Current Spiking                                       
\begin{table}[H]                                                                         
\centering                                                                            
\begin{tabular}{|l|l|l|}                                                              
\hline                                                                                
gM & M-Type K+ Current Curve Observations \\                                           
\hline                                                                                                                                                                
0.500 & Shift to pair spiking \\
\hline                                                                                
0.600 & Pairs of spikes decrease in frequency (spread out) \\
\hline
0.700 & Return to single spiking but with modified waveform \\
\hline                                                                                
1.200 & Spikes are very low frequency (highly spaced out) \\                                                            
\hline                                                                                
1.300 & Spiking is abolished. After the first spike, only a subthreshold transient remains. \\                                                                
\hline                                                                                
\end{tabular}                                                                         
\caption{M-Type K+ Current Curve Spiking Frequencies at the Transition gM values.}
\label{table:2b}                                                                 
\end{table}

\subQuestion{Contributions of the M-type K+ current in spiking frequency}  

\slntParagraph{Comparing the subplots in Figure \ref{Fig:2a1} in addition to the two tables(Table \ref{table:2a3} (Voltage Table) and \ref{table:2b} (Current Table)), we can observe the following: The shift from continuous single spiking to clustered pairs of spikes was observed at $gM = 0.5$ for both the voltage and current plots. Both plots were observed to display a transition back to what qualitatively appeared to be a single-spiking regime with altered spike waveform shape at $gM = 0.7$. At $gM = 1.2$, a period of non-spiking sub-threshold transient is observed after the first spike only for the current (and not the voltage) plot. It was only at the next timestep $gM = 1.3$ that spiking is abolished in favor of subthreshold oscillations in both plots.}

\slntParagraph{The effect of increasing gM on spiking frequency can be observed in Figure \ref{Fig:2a2}. Within a given burst size (number of spikes per burst) the spike frequency decreases monotonically for increasing values of gM. A sharp deviation from this pattern can be observed at values where the number of spikes per burst transitions to a different, such as $gM = 0.5$, and $gM = 0.7$. Consistent with our prior interpretation, spiking disappears entirely for $gM >= 1.3$.}

\subQuestion{Interaction of Ionic Currents}

\slntParagraph{Within the observed region, the M-type K+ current ($\sub{I}{MK+}$) was the primary determinant of the spiking frequency for the membrane voltage. A strong overall trend was observed where increasing $gM$ resulted in a decrease in firing frequency. I suspect that this current serves to repolarize the neuron back to its resting potential after firing. }


%\slntParagraph{
%Up to gM = 0.4 we see the frequency of spikes decrease.\\
%At gM = 0.5, we see a change in dynamics where spikes occur in bursts of two at a time.\\
%At gM = 0.6, we see the pairs of spikes decrease in frequency (spread out) \\
%At gM = 0.7, the spikes have spread out to the extent that it looks more similar to single-spike firing, although the falling edge of the spike waveform remains smooth unlike the sharp cutoffs seen originally. \\
%At gM = 1.2: spikes are very low frequency (highly spaced out) with sprawling looking waveforms with sharp peaks. \\
%At gM = 1.3: spiking is abolished, and after the first spike only a subthreshold transient remains. \\
%As gM is increased further up to $1.5$, the number of identifiable peaks in the subthreshold oscillation decreases. \\
%}

%\begin{table}                
%\centering                   
%\begin{tabular}{|l|l|l|}       
%\hline                       
%gM & Observation & $\text{Freq}_{\text{spike}}$ \\
%\hline                       
%0.5 & shift to pair spiking & 10.381 \\            
%\hline                       
%0.7 & back to single spiking with different waveform & 9.275 \\             
%\hline                       
%1.2 & see beginning die out with transient & 53.171 \\            
%\hline                       
%1.3 & see all die out with only transient & 73.125 \\            
%\hline                       
%\end{tabular}                
%\caption{Table Containing the Relevant Spike Frequencies at the Transition gNaP values.}     
%\label{table:2}   
%\end{table}




% P3
\Question{MATLAB Golomb Neuron Model - Persistent Na+ current and Bursting}

%\begin{figure}[H]
%\centering
%\includegraphics[width=0.9\textwidth]{Results/2-2}
%\caption{\label{Fig:3} Plots for $\mgate{} = 7 \utext{mV}$, $\hgate{} = -9.5 \utext{mV}$, $\ngate{} = 0 \utext{mV}$ used to determine the threshold current amplitude of $0.6 \utext{nA}$ for Nav 1.1.}
%\end{figure}

\begin{figure}[H]
\centering
\includegraphics[width=0.9\textwidth]{Results/3-1}
\caption{\label{Fig:3a1} Membrane Voltage and Persistent Na+ Current plots for several important values of gNaP that mark a transition in firing dynamics. From top to bottom, we have gNaP = {0.06, 0.07, 0.14, 0.2}. }
\end{figure}


\begin{table}[H]                
\centering                   
\begin{tabular}{|c|l|c|}       
\hline                       
gNaP & Observation & $\text{Freq}_{\text{burst}}$ \\
\hline                       
0.060 & Last single (1) spiking regime & 10.381 \\            
\hline                       
0.070 & full pair (2) spiking & 9.275 \\             
\hline                       
0.140 & full triple (3) spiking & 53.171 \\            
\hline                       
0.200 & full quadruple (4) spiking & 73.125 \\            
\hline                       
\end{tabular}                
\caption{Table Containing the Relevant Burst Frequencies at the Transition gNaP values.}     
\label{table:3}   
\end{table}                  

\subQuestion{gNaP and the Number of Spikes per Burst}

\slntParagraph{The gNaP values where the number of spikes per burst change are listed in Table \ref{table:3} and plotted in Figure \ref{Fig:3a1}.}

\subQuestion{Interburst Interval}

\slntParagraph{The burst frequency values are again listed in Table \ref{table:3}. The burst frequency increases as a function of increasing gNaP values. }


\subQuestion{Interaction of Persistent Na+ current and M-type K+ current}

\begin{figure}[H]
\centering
\includegraphics[width=0.9\textwidth]{Results/3-etc-2}
\caption{\label{Fig:3c1} Zoomed in: Membrane Voltage and Persistent Na+ Current plots for several important values of gNaP that mark a transition in firing dynamics. From top to bottom, we have gNaP = {0.06, 0.07, 0.14, 0.2}. }
\end{figure}

\slntParagraph{The increasing gNaP values serve to increase the magnitude of the $\sub{I}{NaP}$, which works to keep the neuron depolarized immediately after spiking, making it more likely that another spike will occur (creating a burst). On the other hand, $\sub{I}{MK+}$ serves to terminate bursting and reset the neuron back to its resting potential so it may burst again more quickly. These dynamics can be seen in Figure \ref{Fig:3c1}.}


\end{document}
